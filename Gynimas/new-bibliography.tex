\begin{thebibliography}{100}

\bibitem{xrayTools} U. Eglė, “Diagnostic X-ray hardware,” 2011, (In lithuanian).

\bibitem{bioInspired} M. Woźniak and D. Połap, “Bio-inspired methods modeled for respiratory disease detection from medical images,” Swarm Evol. Comput., vol. 41, pp. 69–96, Aug. 2018.

\bibitem{lungNodules} M. Woźniak, D. Połap, G. Capizzi, G. Lo Sciuto, L. Kośmider, and K. Frankiewicz, “Small lung nodules detection based on local variance analysis and probabilistic neural network,” Comput. Methods Programs Biomed., vol. 161, pp. 173–180, Jul. 2018.

\bibitem{radiodiagnosis} B. V. Daga, V. R. Shah, and S. V. Daga, Radiodiagnosis, nuclear medicine, radiotherapy and radiation oncology. Jaypee Brothers Medical Publishers, 2013.

\bibitem{cancer} A. Berrington de González et al., “Projected Cancer Risks From Computed Tomographic Scans Performed in the United States in 2007,” Arch. Intern. Med., vol. 169, no. 22, p. 2071, Dec. 2009.

\bibitem{comparison} Y. Y. Kim, H. J. Shin, M. J. Kim, and M.-J. Lee, “Comparison of effective radiation doses from X-ray, CT, and PET/CT in pediatric patients with neuroblastoma using a dose monitoring program.,” Diagn. Interv. Radiol., vol. 22, no. 4, pp. 390–4, 2016.

\bibitem{oecd2018} European Commission and OECD, Health at a Glance: Europe 2018 STATE OF HEALTH IN THE EU CYCLE. 2018.

\bibitem{yeonCT} Y. Y. Kim, H. J. Shin, M. J. Kim, and M.-J. Lee, “Comparison of effective radiation doses from X-ray, CT, and PET/CT in pediatric patients with neuroblastoma using a dose monitoring program.,” Diagn. Interv. Radiol., vol. 22, no. 4, pp. 390–4, 2016.

\bibitem{dataRsna} “RSNA Bone Age data set from Kaggle.” [Online]. Available: https://www.kaggle.com/kmader/rsna-bone-age. [Accessed: 01-Mar-2019].

\bibitem{dataNih} “NIH Chest X-rays data set from Kaggle.” [Online]. Available: https://www.kaggle.com/nih-chest-xrays/data. [Accessed: 01-Mar-2019].

\bibitem{dataMura} “Stanford University MURA Dataset: Towards Radiologist-Level Abnormality Detection in Musculoskeletal Radiographs.” [Online]. Available: https://stanfordmlgroup.github.io/competitions/mura/. [Accessed: 01-Mar-2019].

\bibitem{poisson} S. Lee, M. S. Lee, and M. G. Kang, “Poisson–Gaussian Noise Analysis and Estimation for Low-Dose X-ray Images in the NSCT Domain,” Sensors, vol. 18, no. 4, p. 1019, Mar. 2018.

\bibitem{simulNoise} Hanying Li, T. Toth, S. McOlash, Jiang Hsieh, and N. Bromberg, “Simulating low dose CT scans by noise addition,” in 2002 IEEE Nuclear Science Symposium Conference Record, 2002, vol. 3, pp. 1832–1834.

\bibitem{ssim} Z. Wang, A. C. Bovik, H. R. Sheikh, S. Member, E. P. Simoncelli, and S. Member, “Image Quality Assessment: From Error Visibility to Structural Similarity,” vol. 13, no. 4, pp. 1–14, 2004.

\bibitem{a1} N. Johnston et al., “Improved Lossy Image Compression with Priming and Spatially Adaptive Bit Rates for Recurrent Networks,” Mar. 2017.




% \bibitem{scanParams} S. P. Raman, M. Mahesh, R. V. Blasko, and E. K. Fishman, “CT scan parameters and radiation dose: Practical advice for radiologists,” J. Am. Coll. Radiol., vol. 10, no. 11, pp. 840–846, 2013.
% % NOTE: last updated 0227 12:17


% \bibitem{lowdose} A. Neverauskiene et al., “Image based simulation of the low dose computed tomography images suggests 13 mAs 120 kV suitability for non-syndromic craniosynostosis diagnosis without iterative reconstruction algorithms,” Eur. J. Radiol., vol. 105, pp. 168–174, Aug. 2018.

% \bibitem{ctPrinciples} Jiang Hsieh, Computed Tomography: Principles, Design, Artifacts, and Recent Advances, Second Edition, 2009.

% \bibitem{state} L. L. Geyer et al., “State of the Art: Iterative CT Reconstruction Techniques,” Radiology, vol. 276, no. 2, pp. 339–357, Aug. 2015.

% \bibitem{stacked} P. Vincent, H. Larochelle, I. Lajoie, Y. Bengio, and P.-A. Manzagol, “Stacked Denoising Autoencoders: Learning Useful Representations in a Deep Network with a Local Denoising Criterion,” J. Mach. Learn. Res., vol. 11, pp. 3371–3408, 2010.

% \bibitem{cascaded} D. Wu, K. Kim, G. El Fakhri, and Q. Li, “A Cascaded Convolutional Neural Network for X-ray Low-dose CT Image Denoising,” May 2017.

% \bibitem{denoising} J. Xie, L. Xu, and E. Chen, “Image Denoising and Inpainting with Deep Neural Networks,” pp. 341–349, 2012.

% \bibitem{compression} G. Toderici et al., “Full Resolution Image Compression with Recurrent Neural Networks,” Aug. 2016.

% \bibitem{lungDiseases} Q. Ke et al., “A neuro-heuristic approach for recognition of lung diseases from X-ray images,” Expert Syst. Appl., vol. 126, pp. 218–232, Jul. 2019.



% \bibitem{bakThesis} T. Černius, “Image reconstruction of x-ray images using deep learning,” Vilnius University, 2018, (In lithuanian).

% \bibitem{ultrasound} K. Xu, X. Liu, H. Cai, and Z. Gao, “Full-reference image quality assessment-based B-mode ultrasound image similarity measure,” Jan. 2017.

% \bibitem{seven} D. M. Chandler, “Seven Challenges in Image Quality Assessment: Past, Present, and Future Research,” ISRN Signal Process., vol. 2013, pp. 1–53, 2013.

% \bibitem{wrongMSE} A. B. Watson and Bernd, Digital images and human vision. MIT Press, 1993.

% \bibitem{imgQualityDifficult} Z. Wang, A. C. Bovik, and L. Lu, “Why is image quality assessment so difficult?,” in IEEE International Conference on Acoustics Speech and Signal Processing, 2002, p. IV-3313-IV-3316.

% \bibitem{perceptual} W. Lin and C.-C. Jay Kuo, “Perceptual visual quality metrics: A survey,” J. Vis. Commun. Image Represent., vol. 22, no. 4, pp. 297–312, May 2011.
% % The bottom 5 go together

% \bibitem{deepsim} F. Gao, Y. Wang, P. Li, M. Tan, J. Yu, and Y. Zhu, “DeepSim: Deep similarity for image quality assessment,” Neurocomputing, vol. 257, pp. 104–114, Sep. 2017.

% \bibitem{haar} R. Reisenhofer, S. Bosse, G. Kutyniok, and T. Wiegand, “A Haar Wavelet-Based Perceptual Similarity Index for Image Quality Assessment,” Jul. 2016.

% \bibitem{srsim} L. Zhang and H. Li, “SR-SIM: A fast and high performance IQA index based on spectral residual,” in 2012 19th IEEE International Conference on Image Processing, 2012, pp. 1473–1476.

% \bibitem{gradientSim} W. Xue, L. Zhang, X. Mou, and A. C. Bovik, “Gradient Magnitude Similarity Deviation: A Highly Efficient Perceptual Image Quality Index,” IEEE Trans. Image Process., vol. 23, no. 2, pp. 684–695, Feb. 2014.

% \bibitem{fsim} Lin Zhang, Lei Zhang, Xuanqin Mou, and D. Zhang, “FSIM: A Feature Similarity Index for Image Quality Assessment,” IEEE Trans. Image Process., vol. 20, no. 8, pp. 2378–2386, Aug. 2011.
% % they stop going together


% \bibitem{unet} O. Ronneberger, P. Fischer, and T. Brox, “U-Net: Convolutional Networks for Biomedical Image Segmentation,” May 2015.

% \bibitem{frontier} G. Wang, J. C. Ye, K. Mueller, and J. A. Fessler, “Image Reconstruction is a New Frontier of Machine Learning,” IEEE Trans. Med. Imaging, vol. 37, no. 6, pp. 1289–1296, Jun. 2018.

% \bibitem{dnnIqa} S. Bosse, D. Maniry, K.-R. Muller, T. Wiegand, and W. Samek, “Deep Neural Networks for No-Reference and Full-Reference Image Quality Assessment,” IEEE Trans. Image Process., vol. 27, no. 1, pp. 206–219, Jan. 2018.

% \bibitem{combining} V. V. Lukin, N. N. Ponomarenko, O. I. Ieremeiev, K. O. Egiazarian, and J. Astola, “Combining full-reference image visual quality metrics by neural network,” 2015, p. 93940K.




\end{thebibliography}

%%%%% CLEAR DOUBLE PAGE!
% \newpage{\pagestyle{empty}\cleardoublepage}