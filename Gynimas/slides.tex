\documentclass[aspectratio=169]{beamer}

% \usetheme{CambridgeUS}
% \usecolortheme{default}

% PREVIOUS RED THEME
% \usetheme{Madrid}
% \usecolortheme{beaver}

\setbeamertemplate{frametitle}[default][center] % Centering of slide heading

%%%%%%%%% CUSTOM THEME %%%%%%%%%%
% Found in: https://ramblingacademic.com/2015/12/08/how-to-quickly-overhaul-beamer-colors/

\usetheme{Madrid}
% \useoutertheme{miniframes} % Alternatively: miniframes, infolines, split
\useinnertheme{circles}

% ORIG COLORS
% \definecolor{UBCblue}{rgb}{0.04706, 0.13725, 0.26667} % UBC Blue (primary)
% \definecolor{UBCgrey}{rgb}{0.3686, 0.5255, 0.6235} % UBC Grey (secondary)

\definecolor{UBCblue}{rgb}{0.4823, 0.0, 0.2471} % VU (primary)
\definecolor{UBCgrey}{rgb}{0.8627, 0.8627, 0.8627} % VU (secondary)

%R:123 G:0 B:63

\setbeamercolor{palette primary}{bg=UBCblue,fg=white}
\setbeamercolor{palette secondary}{bg=UBCblue,fg=white}
\setbeamercolor{palette tertiary}{bg=UBCblue,fg=white}
\setbeamercolor{palette quaternary}{bg=UBCblue,fg=white}
\setbeamercolor{structure}{fg=UBCblue} % itemize, enumerate, etc
\setbeamercolor{section in toc}{fg=UBCblue} % TOC sections

% Override palette coloring with secondary
\setbeamercolor{subsection in head/foot}{bg=UBCgrey,fg=white}

% TODO LIST for conference
% 1. unify colours
% 2. use MIF colour palette or VU colour palette
% 3. ...

% \addtobeamertemplate{navigation symbols}{}{%
%     \usebeamerfont{footline}%
%     \usebeamercolor[fg]{footline}%
%     \hspace{1em}%
%     \insertframenumber/\inserttotalframenumber
% }

\usepackage[utf8x]{inputenc}
\def\LTfontencoding{L7x}
\PrerenderUnicode{ąčęėįšųūž}
\usepackage{times}
\usepackage[T1]{fontenc}
\usepackage{color}
\usepackage{verbatim}
\usepackage{graphicx}
\usepackage{fancyvrb}
\usepackage{bm}
\usepackage{amsfonts}
\usepackage{float}
\usepackage{hyperref}
\usepackage{tikz}
\usepackage{textpos} % FOR LOGO
% \usepackage[english]{babel} # THE SLIDES ARE IN LITHUANIAN
\usepackage{LTPlius} % COMMENT IF ENGLISH
\usepackage{tabularx,booktabs}
% \usepackage{enumitem}% http://ctan.org/pkg/enumitem

% set numbers to image, tables, ... captions.
% From: https://tex.stackexchange.com/a/216437
\setbeamertemplate{caption}[numbered]{}% Number float-like environments

\usepackage{caption} % <<< does not include "figure ..."
% \usepackage[labelformat=empty]{caption} % <<< includes "figure ..."
%\usepackage{natbib}

\usepackage{subfig}

\providecommand{\btVFill}{\vskip0pt plus 1filll}

% \newcommand{\todo}[1]{} % uncomment this before publishing
\newcommand{\todo}[1]{\textcolor{red}{//\textbf{TODO: {#1}}} }

\newcommand{\eng}[1] {(angl. \textit{#1})}
\newcommand{\engp}[2] {(angl. \textit{#1}, toliau: #2)}

\usetikzlibrary{matrix,shapes,arrows,fit,tikzmark}
\newcolumntype{.}{D{.}{.}{-1}}

% Some options common to all the nodes and paths
\tikzset{   
        every picture/.style={remember picture,baseline},
        every node/.style={anchor=base,align=center,outer sep=1.5pt},
        every path/.style={thick},
        }

\newcommand\marktopleft[1]{%
    \tikz[overlay,remember picture] 
        \node (marker-#1-a) at (.1em,.3em) {};%
}
\newcommand\markbottomright[1]{%
    \tikz[overlay,remember picture] 
        \node (marker-#1-b) at (.1em,.3em) {};%
    \tikz[overlay,remember picture,inner sep=3pt]
        \node[draw=red,rounded corners,fit=(marker-#1-a.north west) (marker-#1-b.south east)] {};%
}

\newcolumntype{P}[1]{>{\centering\arraybackslash}p{#1}}

% TODO: attempt to remove the 1s
% TITLE PAGE 

\title[Privačios informacijos išsaugojimas]
{Privačios informacijos išsaugojimas taikant dirbtinio intelekto technologijas}
% \subtitle{Presentation}

\author[Paulius Milmantas]
{Paulius Milmantas\\
Darbo vadovas: dr. Linas Petkevičius}
 
\institute[VU-MIF] % (optional)
{
  Vilniaus Universitetas \\
  Matematikos ir informatikos fakultetas \\
}

\date[2021-05-14]{Bakalauro darbo gynimas}


% Let's get started
\begin{document}

\addtobeamertemplate{frametitle}{}{%
\begin{textblock*}{100mm}(.95\textwidth,-0.95cm)
\includegraphics[height=1cm,width=1cm]{VU_Logo_pilkas.png}
\end{textblock*}
}

%
\begin{frame}
  \titlepage
\end{frame}
%

% Tiriama sritis
 \begin{frame}[c]{Tiriama sritis}
 
\setlength{\parindent}{10ex} Mašininis mokymas yra dirbtinio intelekto sritis, kuri pasitelkia statistinius algoritmus, kad apibrėžtų duomenų generavimo mechanizmą, ar egzistuojančius sąryšius, priklausomybes. 

 \end{frame}
% Tikslas ir uždaviniai
\begin{frame}[c]{Tikslas ir uždaviniai}

\par {\bf Darbo tikslas} - ištirti ir palyginti privatumą saugančius dirbtinio intelekto algoritmus pagal jų saugumą, našumą ir panaudojamumą, bei pateikti rekomendacijas.
\par \\~\\ Darbo tikslui įgyvendinti, iškelti šie {\bf uždaviniai}:
\begin{enumerate}
	\item Išanalizuoti esamus algoritmus pagal jų saugumą ir panaudojamumą.
	\item Identifikuoti kriterijus, kurių pagalba galima įvertinti privatumo išsaugojimą, bei palyginti algoritmus tarpusavyje.
	\item Ištirti kurie algoritmai yra realizuoti ir realizuoti dalį algoritmų, kurie nėra atvirai prieinami.
	\item Palyginti algoritmus pagal našumą ir pateikti rekomendacijas.
\end{enumerate}

\end{frame}
% Problematika
\section{Problematika}

 \begin{frame}[c]{Problematika}

    \begin{enumerate}
        \item Turint sukurtą modelį, neturi būti galima atgaminti duomenų, pagal kuriuos jis buvo mokomas, bei negali būti identifikuoti asmenys [1]. 
        \item Trečios šalys neturi matyti įvedamų duomenų. Tai gali būti tinklo saugumo spragos, duomenų surinkimo aplikacijų spragos ir t.t…
        \item Modelio išvesties neturi matyti asmenys, kuriems šie duomenys nepriklauso.
        \item Sukurtas modelis negali būti niekieno pasisavintas. 
    \end{enumerate}

 \end{frame}
% Modelių duomenų lyginimas (1)
\begin{frame}[c]{Modelių duomenų lyginimas (1)}
 
 \begin{equation}
     atvirumas(s[r])_{\theta} = log_{2}|r| - log_{2} rangas_{\theta}(s[r])
 \end{equation}
 
Naudojama teorijoje, dėl sunkiai apskaičiuojamo rango [2].

$\bf s$ - duomenų rinkinys. \\
$\bf r \in R$, parenkamas atsitiktinai.

\end{frame}
% Modelių duomenų lyginimas (2)
\begin{frame}[c]{Modelių duomenų lyginimas (2)}

\begin{equation}
    atvirumas(s[r])_{\theta} = -log_{2} \int_{0}^{Px_{\theta}(s[r])} \rho(x)dx
\end{equation}

Dėl grafinės interpretacijos naudojama praktikoje.

$\bf Px$ - logaritminis entropijos matas. \\
$\bf s[r]$ - entropija yra $\rho(.)$ pasiskirstymo distribucijos.

\end{frame}
% Tyrimo metodika
\begin{frame}[c]{Pasiūlyta tyrimo metodika}

\begin{equation}
    DMDK = {\sum_{n=0}^{m} ({\sum_{k=0}^{h} (max((|\epsilon| + D_{n, k}) : \epsilon \in R))}/{h})}/{m}
\end{equation}

$\bf DMDK$ - Didžiausias maksimalus duomenų nuokrypis. \\
$\bf D_{eilut:n,stulp:k}$ - duomenys n eilutėje ir k stulpelyje. \\
$\bf \epsilon$ - ieškomas didžiausias galimas kintamasis, su kuriuo modelis nepakeičia išvesties rezultatų. \\
$\bf m$ - duomenų eilučių skaičius. \\
$\bf h$ - parametrų skaičius (stulpeliai).

\end{frame}
% ---- Metrikos validavimą pašalinu, dėl laiko trūkumo.
% Metrikos validacija (1)
%\begin{frame}[c]{Metrikos validavimas (1)}
 
Pagal KMI ir gimdymų skaičių prognozuojama, ar moteris serga cukriniu diabetu.

\begin{itemize}
    \item Kai modelio DMDK yra mažas –- gauti duomenys eksperimente buvo artimi pradiniams duomenims. 
    \item Kai modelio DMDK yra didelis – nepavyko gauti panašių duomenų.
\end{itemize}



\end{frame}
% Metrikos validacija (2)
%\begin{frame}[c]{Metrikos validavimas (2)}
 
\begin{itemize}
    \item Eilučių duomenys buvo padauginti iš N. Sukūrus kelis naujus modelius su skirtingais N, DMDK reikšmė išlieka panaši.
    \item Tarkime, kad pirmas modelis turi viena parametrą - KMI. Pagal šį parametrą, modelis prognozuoja, ar žmogus serga cukriniu diabetu ar ne. Modelio tikslumas yra 54\%, jis visą laiką prognozuoja, kad žmogus serga cukriniu diabetu. DMDK reikšmė artėja link begalybės.
\end{itemize}

\end{frame}
% Pridėto triukšmo tyrimas
\begin{frame}[c]{Pridėto triukšmo tyrimas}
 
\begin{figure}[ht]%
 \centering
 \subfloat[Tikslumo tyrimas]{\includegraphics[width=7cm,height=5cm,keepaspectratio]{img/tr_tyr_acc.png}\label{fig:Tikslumo tyrimas}}%
 \subfloat[DMDK tyrimas]{\includegraphics[width=7cm,height=5cm,keepaspectratio]{img/tr_tyr_dmdk.png}\label{fig:DMDK tyrimas}}%
\end{figure}

\end{frame}
% Pallier tyrimas
\begin{frame}[c]{Pallier tyrimas}
 
\begin{figure}[ht]%
 \centering
 \subfloat[DMDK tyrimas]{\includegraphics[width=7cm,height=5cm,keepaspectratio]{img/pal_tyr_1.png}\label{fig:DMDK tyrimas}}%
 \subfloat[Tikslumo tyrimas]{\includegraphics[width=7cm,height=5cm,keepaspectratio]{img/pal_tyr_2.png}\label{fig:Tikslumo tyrimas}}%
\end{figure}

\end{frame}
% PyTorch karkaso neuroninio tinklo tyrimas
\begin{frame}[c]{PyTorch karkaso neuroninio tinklo tyrimas}
 
\begin{figure}[ht]%
 \centering
 \subfloat[Nuostolių f-jos priklausomybės tyrimas]{\includegraphics[width=7cm,height=5cm,keepaspectratio]{img/pap_tyr_1.png}\label{fig:DMDK tyrimas}}%
 \subfloat[Tikslumo tyrimas]{\includegraphics[width=7cm,height=5cm,keepaspectratio]{img/pap_tyr_2.png}\label{fig:Tikslumo tyrimas}}%
\end{figure}

\end{frame}
% Rezultatų aprobavimas
\begin{frame}[c]{Rezultatų aprobavimas}
 
\par Baigiamojo darbo rezultatai buvo pristatyti nacionalinėje konferencijoje, o pranešimo tezės publikuotos konferencijos leidinyje:

\item {\bf Paulius Milmantas} (2021) {\it Privačios informacijos išsaugojimas taikant dirbtinio intelekto technologijas}, Vilnius University Open Series, pp. 71-76. doi: 10.15388/LMITT.2021.8.

\end{frame}
% Išvados
\begin{frame}[c]{Išvados}
 
\begin{itemize}
    \item Esant aukštam modelio tikslumui, rekomenduojama naudoti homomorfinį šifravimą. Esant mažesniam, nei 70\% tikslumui, kai modelis priima < 20 parametrų, rekomenduojama naudoti PyTorch karkaso neuroninius tinklus.
    \item Esant didesniam modelio parametrų skaičiui, PyTorch karkaso neuroniniai tinklai labiau prisimena pradinius mokymosi duomenis ir juos galima lengviau atskleisti.
    \item Naudojant neuroninius tinklus be homomorfinio šifravimo ir modelio tikslumui esant daugiau nei 80\%, rekomenduojama pridėti triukšmą prie pradinių modelio duomenų.
    \item Pradinių duomenų kiekis neturi įtakos modelio duomenų saugumui.
\end{itemize}

\end{frame}
% Šaltiniai
\begin{frame}[c]{Šaltiniai}
 
\\\ \textbf{[1]} Patricia Thaine. Perfectly privacy-preserving ai, 01 2020 
\\\ \textbf{[2]} Nicholas Carlini, Chang Liu, Ulfar Erlingsson, Jernej Kos, and Dawn Song.   The secretsharer: Evaluating and testing unintended memorization in neural networks, 2019.

\end{frame}

% \appendix

\end{document}
