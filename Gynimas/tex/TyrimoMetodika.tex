\begin{frame}[c]{Pasiūlyta tyrimo metodika}

\begin{equation}
    DMDK = {\sum_{n=0}^{m} ({\sum_{k=0}^{h} (max_{\epsilon}((|\epsilon| + D_{n, k}) : \epsilon \in R, modelis(|\epsilon| + D_{n, k}) = modelis(D_{n, k})))}/{h})}/{m}
\end{equation}

$\bf DMDK$ - Didžiausias modelio duomenų nuokrypis. \\
$\bf D_{eilut:n,stulp:k}$ - duomenys n eilutėje ir k stulpelyje. \\
$\bf \epsilon$ - ieškomas didžiausias galimas kintamasis, su kuriuo modelis nepakeičia išvesties rezultatų. \\
$\bf m$ - duomenų eilučių skaičius. \\
$\bf h$ - parametrų skaičius (stulpeliai).

\end{frame}